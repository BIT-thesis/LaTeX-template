%%==================================================
%% ch2.tex for BIT Master Thesis
%% modified by yang yating
%% version: 0.5
%% last update: April 25th, 2017
%%==================================================

\chapter{模板使用}
\label{chap:textStructure}

本章的目的是介绍\LaTeX{}的文本控制流程,介绍学位论文中各章节分布以及模板内部组成,以及章节内的交叉引用问题,用户可以根据自身对\LaTeX{}的熟悉程度适当地略过阅读。在了解了本章的内容后,用户即可通过文本内容的粘贴和复制,快速实现生成一个格式满足基本需求的学位论文。

以硕士模板BIT-thesis-template-grd为例,文件布局如图 \ref{layout-master} 所示。 

\begin{lstlisting}[basicstyle=\small\ttfamily,caption={BIT-thesis-template-grd 模板文件布局},label=layout-master,numbers=none]
  ├── demo.tex              主控文件
  ├── demo.pdf              生成的论文pdfBIT-thesis-template-grd
  ├── BIT-thesis-grd.cls    格式控制文件
  ├── GBT7714-2005NLang.bst 参考文献格式控制文件
  ├── chapters              章节文件夹
  │   ├── abstract.tex       摘要
  │   ├── chapter01.tex      第一章
  │   ├── conclusion.tex     总结
  │   ├── app1.tex           附录A
  │   ├── pub.tex            攻读学位期间发表论文与研究成果清单
  │   └── thanks.tex         致谢
  ├── figures               图片文件夹
  │   └── figure1.png   
  ├── reference             参考文献文件夹
  │   └── chap1.bib
  ├── BIT-thesis-run.cmd    运行脚步cmd
  └── BIT-thesis-run.sh     运行脚本sh
  
      
\end{lstlisting}

\section{认识模板组成}

\subsection{格式控制文件}
\label{sec:format}

格式控制文件控制着论文的表现形式,包括以下两个文件:BIT-thesis-grd.cls 和 GBT7714-2005NLang.bst。
其中,``.cls''控制论文主体格式,``.bst''控制参考文献条目的格式,

一般用户可以``忽略''格式控制文件的存在。因为该文件已经按照《北京理工大学博士、硕士学位
论文撰写规范》进行了修改。有其他格式修改的需要,参见第\ref{sec:thesisformat}章。


\subsection{主控文件~demo.tex}
\label{sec:demotex}

主控文件~demo.tex~的作用就是将分散在多个文件中的内容``整合''成一篇完整的论文。
使用这个模板撰写学位论文时,学位论文内容和素材会被``拆散''到各个文件中:
譬如各章正文、各个附录、各章参考文献等等。
在~demo.tex~中通过``include''命令将论文的各个部分包含进来,从而形成一篇结构完成的论文。
封面页中的论文标题、作者等中英文信息,也是在~demo.tex~中填写。也可以在demo.tex中按照自己的需要引入一些的宏包
\footnote{一般只有当你需要在文档中使用那个宏包时,才需要在导言区中用~usepackage~引入该宏包。如若不然,通过usepackage引入一大堆不被用到的宏包,必然是一场灾难。由于一开始没有一致的设计目标,\LaTeX~ 的各宏包几乎都是独立发展起来的,因重定义命令导致的宏包冲突屡见不鲜。}。

大致而言,在主控文件~demo.tex~中,只需要留意文章有哪些章节以及各章参考文
献内容,不需要具体关注每一章里面的具体内容。需要注意,处理文档时所有的操作命令
{}\cndash{}xelatex, bibtex等,都是作用在~demo.tex~上,而\emph{不是}后面这
些``分散''的文件,请参考\ref{sec:process}小节。

若使用\textbf{硕士论文模板},请在~demo.tex~中~\verb|\documentclass|~命令采用~master~选项;若使用\textbf{博士论文模板},请使用~doctor~选项。同理,单页打印使用~oneside~选项,双面打印使用~twoside~选项。

文章各部分安排如下:
\begin{itemize}
\item ~\verb|\maketitle|~ : 中文封面
\item ~\verb|\makeInfo|~ : 中文信息
\item ~\verb|\makeEnglishInfo|~ : 英文信息
\item ~\verb|\makeVerticalTitle|~ : 打印竖排论文题目
\item ~\verb|\makeDeclareOriginal|~ : 论文原创性声明和使用授权
\item ~\verb|%%==================================================
%% abstract.tex for BIT Master Thesis
%% modified by yang yating
%% version: 0.2
%% last update: Feb 16th, 2017
%%==================================================

\begin{abstract}

  这是一篇BIT-thesis使用指南......
  
  本使用指南......

  \keywords{北京理工大学 \quad 德以明理,学以精工}
\end{abstract}

\begin{englishabstract}

  this is english abstract

  \englishkeywords{BIT, master thesis, XeTeX/LaTeX template}
\end{englishabstract}
|~ :摘要
\item ~\verb|\tableofcontents|~ :加入目录
\item ~\verb|\listoftables|~ :加入表格索引
\item ~\verb|\listoffigures|~ : 加入插图索引
\item ~\verb|\include{chapters/chapter1}|~ : 各章正文内容
\item ~\verb|%%==================================================
%% conclusion.tex for BIT Master Thesis
%% modified by yang yating
%% version: 0.2
%% last update: Feb 16th, 2017
%%==================================================

\chapter*{全文总结\markboth{全文总结}{}}
\addcontentsline{toc}{chapter}{全文总结}

这里是全文总结内容。

|~ : 结论
\item ~\verb|\bibliography{reference/chap1}|~ :参考文献
\item ~\verb|\include{chapters/app1} |~ : 附录
\item ~\verb|\include{chapters/pub}|~ : 攻读学位期间发表论文与研究成果清单
\item ~\verb|%%==================================================
%% thanks.tex  for BIT Master Thesis
%% modified by yang yating
%% version: 0.2
%% last update: April 27th, 2017
%%==================================================


\begin{thanks}


感谢计算机学院2016级研究生杨雅婷对模板的修改更新做了大量工作。由衷感谢在Github对该项目上提出大量珍贵修改意见的老师和同学们。

本项目得到研究生院学位与学部办公室和学生事务中心的资助支持。

\end{thanks}
|~ : 致谢
\item ~\verb|%%==================================================
%% thanks.tex for BIT Master Thesis
%% modified by yang yating
%% version: 0.1
%% last update: Feb 26th, 2017
%%==================================================

\begin{resume}

本人……。

\end{resume}
|~ : 作者简介(博士论文需要)
\end{itemize}


\subsection{论文主体文件夹chapters}
\label{sec:thesisbody}

这一部分是论文的主体,是以``章''为单位划分的。

正文前部分(frontmatter):中英文摘要(abstract.tex)。其他部分,诸如中英文信息封
面、授权信息等,都是根据~demo.tex~所填的信息自动生成好了,不需要单独编写文件。

正文部分(mainmatter):是各章内容,在chapter文件夹中。

正文后的部分(backmatter):附录(app\emph{xx}.tex);致谢(thanks.tex);攻读
学位论文期间发表的学术论文目录(pub.tex);作者简介(resume.tex)。参考文献列
表是自动生成的,也不需要作为一个单独的文件。另外,学校的硕士研究生学位论文模
板没有要求加入作者简介,但\textbf{博士的学位论文要求加入作者简介}。


\subsection{图片文件夹~figures}
\label{sec:figuresdir}

figures~文件夹放置了需要插入文档中的图片文件(PNG/JPG/PDF/EPS)。如果图片较多,建议按章再
划分子目录存储图片。

\subsection{参考文献数据库文件夹~reference}
\label{sec:bibdir}

reference~文件夹放置的是各章``可能''会被引用的参考文献文件。参考文献的元
数据,例如作者、文献名称、年限、出版地等,会以一定的格式记录在纯文本文
件.bib中。最终的参考文献列表是BibTeX处理.bib后得到的,名为~demo.bbl。将参
考文献按章划分的一个好处是,可以在各章后生成独立的参考文献,不过,现在看
来没有这个必要。关于参考文献的管理,可以进一步参考第\ref{chap:example}章
中的例子。


\section{进行论文写作}
\label{sec:format}

本节介绍使用论文模板,修改论文信息、摘要、关键字,以及编辑章节等。

\subsection{封面和标题}
在主控文件demo.tex中填写论文的相应信息。

中文封面信息:
\begin{itemize}
\item 中图分类号(\verb|\classification{TQ028.1}|)
\item UDC分类号(\verb|\UDC{540}|)
\item 论文标题(\verb|\title{论文标题}|)
\item 作者姓名(\verb|\author{姓名}|)
\item 学院名称(\verb|\institute{学院名称}|)
\item 指导教师(\verb|\advisor{教授姓名}|)
\item 答辩委员会主席(\verb|\chairman{教授姓名}|)
\item 申请学位(\verb|\degree{学位名称}|)
\item 学科专业(\verb|\major{专业名称}|)
\item 学位授予单位(\verb|\school{北京理工大学}|)
\item 论文答辩日期(\verb|\defenddate{答辩日期}|)
\end{itemize}


英语封面信息:
\begin{itemize}
\item English Title(\verb|\englishtitle{论文标题}|)
\item Candidate Name(\verb|\englishauthor{姓名}|)
\item School or Department(\verb|\englishinstitute{学院名称}|)
\item Faculty Mentor(\verb|\englishadvisor{教授姓名}|)
\item Chair,Thesis Committee(\verb|\englishchairman{教授姓名}|)
\item Degree Applied(\verb|\englishdegree{学位名称}|)
\item Major(\verb|\englishmajor{专业名称}|)
\item Degree by(\verb|\englishschool{Beijing Insititute of Technology}|)
\item The Data of Defence(\verb|\englishdate{答辩日期}|)
\end{itemize}

\subsection{摘要和关键字}
摘要内容,放置于chapters文件夹下的abstract.tex中,

中文摘要于\verb|abstract|环境编写;\verb|\keywords|为中文关键字。英文摘要于\verb|englishabstract|环境编写;\verb|\englishkeywords|为中文关键字。

\begin{lstlisting}[language={[LaTeX]TeX}, caption={ 中英文摘要 }]
\begin{abstract}
	本文 ...
\keywords { 形状记忆;聚氨酯 }
\end{abstract}

\begin{englishabstract}
   In order to exploit ...
\englishkeywords{shape memory properties; polyurethane}
\end{englishabstract}
\end{lstlisting}


\subsection{正文章节}
章节的设置分别通过关键字完成,按照章节的级别依此如表~\ref{tab:setSection}所示,关于文档中具体章节的关键词设置可以参看原宏包中tex文件夹下的实例文件。

\begin{table}[htb]
 \centering
  \caption{章节设置关键字}     % title of Table
  \label{tab:setSection}    % label of Table
  \begin{tabular}{cl}
    \hline
    章节级别        & 关键字     \\
    \hline
     章        & \verb|\chapter| \\
     节        & \verb|\section | \\
    子节      & \verb|\subsection |\\
    表格名称       & \verb|\caption{标题名称}| \\
    引用标签       & \verb|\label{引用名称}| \\
    \hline
  \end{tabular}
\end{table}

\subsection{其他部分}


全文总结、攻读学位期间发表论文与研究成果清单、致谢的内容,均位于chapters文件夹下,分别为conclusion.tex,pub.tex,thanks.tex。

\section{交叉引用}
\subsection{公式、图表和插图引用}
\label{sec:refofFigAndTab}
交叉引用的前提是需要在定义章节、公式和图表的时候都对其进行命名标签(即\textbackslash label\{sec:labelName\}命令),在实际使用过程中通过标签进行引用。根据引用的特点可以将应用分成表~\ref{tab:citeType}中所示三类。

\begin{table}[htb]
 \centering
  \caption{交叉引用类型}       % title of Table
  \label{tab:citeType}    % label of Table
  \begin{tabular}{cl}
    \hline
    引用类型     & 关键字     \\
    \hline
    标签设置        & \textbackslash label\{marker\}  \\
    引用代号        & \textbackslash ref\{marker\}    \\
    引用页码        & \textbackslash pageref\{marker\} \\
    引用文献        & \textbackslash cite\{regLabel\} \\
    \hline
  \end{tabular}
\end{table}

其中,表格和图片的摆放位置由 \textbackslash begin\{table\}或\textbackslash begin\{figure\}后面的中括号设置,例如[htb]表示可以将图表放在当前位置(here)、页面顶端(top)或者页面底端(bottom)。

{\bf{实例1:}}这里是对表格《交叉引用类型》的引用——表~\ref{tab:citeType}位于第~\pageref{tab:citeType}页,其标签为\textbackslash label\{tab:citeType\}。


\subsection{文献引用}
\label{sec:citeRefs}

BIT-Thesis论文模板使用BibTeX处理参考文献。

参考文献的具体内容就是reference文件夹下的chap\textit{xx}.bib,参考文献的元数据(名称、作者、出处等)以一定的格式保存在这些纯文本文件中。
.bib文件也可以理解为参考文献的``数据库'',正文中所有引用的参考文件条目都会从这些文件中``析出''。

正文中引用参考文献时,用\verb+\upcite{...}+可以产生“上标引用的参考文献”,
如\upcite{chen2007act},\upcite{Meta_CN,chen2007act,DPMG}。

具体使用方法参见第\ref{sec:reference}节。
